\documentclass[12pt, letterpaper]{article}

\usepackage[utf8]{inputenc}
\usepackage[spanish, es-tabla]{babel}
\usepackage{geometry}       % Configuración de márgenes
\usepackage{fancyhdr}       % Encabezados y pies de página personalizados
\usepackage{tocloft}        % Personalización de la tabla de contenidos
\usepackage{lastpage}       % Referencia a la última página
\usepackage{hyperref}
\usepackage{graphicx}

% Configuración de márgenes
\geometry{
    left=1.5cm,
    right=1.5cm,
    top=1.5cm,
    bottom=1.5cm,
    headheight=14.5pt
}

\graphicspath{{media/}}

\pagestyle{fancy}
\fancyhf{}
\fancyhead[L]{Manual de Usuario - Tlacuia GCL}
\fancyfoot[C]{Página \thepage\ de \pageref{LastPage}}
\renewcommand{\headrulewidth}{0.4pt}
\renewcommand{\footrulewidth}{0.4pt}

% Información de la portada
\title{
    \vspace*{3cm}
    \textbf{Manual de usuario}\\[1cm]
    \Large\textbf{Tlacuia: Gestor de Concesiones para Librerias}\\[0.5cm]
    \large Versión: 0.3.7
}
\author{
    Autor: Leonardo J.\\
    No. de Revisión: 0
}
\date{
    14 de Febrero del 2024
}


\begin{document}

% Portada
\maketitle
\thispagestyle{empty} % Elimina el número de página en la portada
\newpage

% Tabla de contenidos
\tableofcontents
\newpage

% Contenido del manual
\section{Introducción}

\begin{center}
    \includegraphics[width = 5cm]{TlacuiaLogo.png}
\end{center}

¡Hola personita! Se bienvenido al manual de usuario de Tlacuia Gestor de Concesiones para Librerías, o Tlacuia GCL. Esta es una
aplicación de escritorio desarrollada en Python y diseñada especificamente para el manejo local de datos en librerias mexicanas.  

Este documento te guiara a travez de las funciones basicas y avanzadas de la aplicación, para ayudarte asi a sacarle el mayor 
provecho. En esta version, 0.3.7, despues de mucho trabajo en casi dos semanas y una gran cantidad de modificaciones libero por 
fin la primera de dos herramientas planeadas, este es el Gestor de Concesiones. Este nace siguiendo las especificaciones y 
necesidades de una librería con personitas que quiero y aprecio mucho, Somos Voces en la Zona Rosa de Ciudad de Mexico, 
espero y puedan tener la oportunidad de visitarla algun dia.  

El nombre Tlacuia, proviene del nahuatl y significa “tomar prestado”, esto debido a la naturaleza que suele tener una concesion 
dentro de los sistemas de librerias, reflejando uno de los propositos de esta aplicación. El diseño de Tlacuia busca centralizar 
todo lo relacionado con los documentos de una concesión de librería dentro del Gestor de Concesiones, ayudando al librero o 
administrador atareado a revisar la congruencia de datos entre facturas y otros documentos, como los reportes generados por 
sistemas comunes (por ejemplo, GESLIB). Ademas esta el Asistente de Bodega, esta es una herramienta diseñada facilitar la compleja 
gestión que es mantener un control preciso de libros o articulos para aquellos que emiten concesiones y necesitan registrar 
continuamente el movimiento de sus elementos en distintos espacios físicos, que buscan relacionarlos a través de un documento 
de concesión específico o simplemente poder saber con mayor facilidad el status de los elementos en sus bodegas.  \footnote{
    Actualmente, la herramienta mencionada, Asistente de Bodega, se encuentra en desarrollo y no está integrada en la versión
    0.3.7 de Tlacuia GCL. Se tiene previsto que esta funcionalidad esté disponible en la versión 0.4.x
}


\newpage
\subsection{Requisitos del sistema}
Tlacuia GCL es una aplicación completamente programada en Python, con una interfaz grafica de usuario Qt. Esta posee dos tipos de herramientas, 
convencionales e impulsadas por modelos de lenguaje de gran tamaño de IA, es decir LLM. Ningun dato es procesado fuera de la computadora donde
se esta ejecutando Tlacuia, por lo que según el uso, pueden ser necesarios algunos requisitos minimos para poder aprovecharlo. 

Estos requisitos estan divididos según el tipo de usuario esperado y la arquitectura de su sistema.\footnote{
    Para ARM, Tlacuia solo se compila para procesadores Apple Silicone M1 en adelante. 
}

\subsubsection{Equipos x86 64}
\textbf{Requisitos minimos para usuarios que solo ocuparan las herramientas convencionales}
\begin{itemize}
    \item Procesador: Intel Core i3 (octava generacion o superior) o AMD Ryzen 3 (segunda generacion o superior)
    \item 4 GB
    \item 20 GB (SSD Recomendado)
    \item GPU: Graficos integrados del procesador
    \item SO: Windows 10 o Ubuntu 22.04
\end{itemize}

\textbf{Requisitos minimos para usuarios que ocuparan las herramienta convencionales e impulsadas por IA}
\begin{itemize}
    \item Procesador: : Intel Core i5 (decima generacion o superior) o AMD Ryzen 5 (tercera generacion o superior)
    \item RAM: 16 GB
    \item 50 GB (SSD NVMe Recomendado)
    \item Graficos integrados del procesador o NVIDIA GTX 1660 (o superior) \footnote{
        El modelo LLM utilizado es un destilado de Mistral con 7 mil millones de parametros, este puede ejecutarse sobre la CPU. 
    }
    \item SO: Windows 11 o Ubuntu 22.04
\end{itemize}

\textbf{Requisitos ideales para utilizar todas las herramientas integradas}
\begin{itemize}
    \item Intel Core i7 (onceava generacion o superior) o AMD Ryzen 7 (quinta generacion o superior)
    \item 32 GB
    \item 50 GB (SSD NVMe Recomendado)
    \item GPU: Graficos: NVIDIA RTX 3060 (o superior)
    \item SO: Windows 11 o Ubuntu 22.04
\end{itemize}

\textbf{Equipos ARM}
Tlacuia solo es compatible con computadoras Apple Silicon, para usuarios que utilicen solo las herramientas convencionales cualquier procesador 
M1 o superior debera funcionar perfectamente, para los usuarios que utilicen herramientas de IA se recomiendan procesadores M1 Pro, Max o superior. 

\section{Instalación}
Explica aquí los pasos necesarios para instalar la aplicación. Puedes incluir subsecciones si es necesario.

\section{Uso de la aplicación}
Aquí puedes describir las funcionalidades principales de la aplicación y cómo utilizarlas.

\subsection{Interfaz de usuario}
Explica cómo navegar por la interfaz de usuario y qué significan los diferentes elementos.

\subsection{Funcionalidades clave}
Detalla las funciones más importantes de la aplicación.

\end{document}